\documentclass[11pt,twoside,a4paper]{article}

\begin{document}
\section{Scan\_For\_Matches}
Scan\_for\_matches is a string-searching tool used mainly by biologists to 
search through files of transcribed genetic material, like RNA strings in 
chromosomes. It was developed by Ross Overbeek, David Joerg and Morgan Price, 
and it was written in C. It uses a brute-force approach to solving its task.\\
The reason why an alternative to scan\_for\_matches is looked after is because 
of how it's been coded. The code is difficult to understand and navigate, 
which makes it difficult to modify for additional functionality, something the 
students and researchers at KU's microbiology department desires.\\
In order to reimplement scan\_for\_matches, we must first understand how 
it works. We will first talk about what kind of patterns scan\_for\_matches can 
search for, after which we will talk about how it matches patterns.
\subsection{Patterns}
To describe what a pattern is, we can first look at the following example:
$$p1=3..5 1..3 ~p1$$
Above we first describe a string of a length between 3 and 5, and call that 
p1. We next say that there must be a sequence of characters of length between 1 and 3, 
which must be followed by the reverse complement of p1. This allows users to 
search for sequences of a specific structure. For example, the example would 
match a hairpin loop. One can also specify exactly what kind of sequence 
they would want to match. If we want to specify what must be looped in the earlier 
example, we can write $$p1=3..5 AG ~p1$$
It is also possible to perform approximate matching. This is done by appending 
to a pattern \[m,d,i\], where m is the number of mismatches, d is the number 
of deletions and i is the number of insertions. Using the previous example, 
if we want the looped sequence to allow for one mismatch, we can write 
$$p1=3..5 AG[1,0,0] ~p1$$
\subsection{Flaws}
While we were testing TRE, we discovered that TRE, though slower than scan\_for\_matches, 
found more hits. This means that throughout scan\_for\_matches' lifetime, 
there has been times where scan\_for\_matches couldn't give all matches. 
\end{document}
