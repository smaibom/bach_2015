\documentclass[11pt,twoside,a4paper]{article}
\usepackage{pbox}
\usepackage{amsthm}
\newtheorem{definition}{Definition}
\newtheorem{example}{Example}
\begin{document}
\section{Scan\_For\_Matches}

Scan\_for\_matches is a string-searching tool created by Ross Overbeek, David 
Joerg and Morgan Price in C which searches through FASTA text files. Users specify 
what they wish to search for by defining a pattern, and scan\_for\_matches 
returns all matches that corresponds to the specified pattern.
\begin{definition}\label{patd}
A pattern is defined as follows:\\
\begin{tabular}{|r|l|}
\hline
{\tt ACUG}&Match the sequence ACUG\\
\hline
{\tt 1...5}&Match 1 to 5 characters\\
\hline
{\tt 3...3}&Match exactly 3 characters\\
\hline
{\tt p1=1...5}&Match 1 to 5 characters, and call the sequence p1\\
\hline
{\tt p1 $|$ p2}&Match either p1 or p2\\
\hline
{\tt p1[1,0,0]}&Match p1, allowing for one mismatch\\
\hline
{\tt p1[0,1,0]}&Match p1, allowing for one deletion\\
\hline
{\tt p1[0,0,1]}&Match p1, allowing for one insertion\\
\hline
{\tt length(p1+p2) $<$ 5}&The combined length of p1 and p2 must not exceed 4\\
\hline
{\tt r1=\{AB, BA\}}&\pbox{20cm}{Create a pattern rule where A is the complement of B, \\and B is the complement of A, and call it r1}\\
\hline
{\tt $<$p1}&Match the reverse of p1\\
\hline
{\tt \textasciitilde p1}&\pbox{20cm}{Match the reverse complement of p1 using the G-C, \\C-G, A-T and T-A pairing rule}\\
\hline
{\tt r1\textasciitilde p1}&\pbox{20cm}{Match the reverse complement of p1 using r1 rules}\\
\hline
{\tt \textasciicircum ~p1}&\pbox{20cm}{Match only p1 if it is at the start of a string}\\
\hline
{\tt p1 \$}&Match only p1 if it is at the end of a string\\
\hline
\end{tabular}
\end{definition}

\begin{definition}\label{patc}
Let {\tt E} be any pattern that's in the alphabet $\Sigma$ as defined in Definition \ref{patd}. 
Let $\epsilon$ be the empty string.
Let {\tt A} be a string that we are processing to see if the pattern is valid.
A pattern may then be constructed as such: \begin{center}
{\tt A = A* A | $\epsilon$}\\
{\tt A* = E}\end{center}
\end{definition}
Definition \ref{patc} states that a pattern may be any combination of the alphabet 
defined in \ref{patd}.
Using these patterns, it is possible to make very specific or very broad 
searches in a text file. 

\begin{example}
Say we want to write a pattern that finds the sequence {\tt GUUC}, allowing 
one mismatch, followed by a random sequence which has a length between 3 and 5, 
followed by the reverse complement of the first sequence that we found. We can 
write this as \begin{center}
{\tt p1=GUUC[1,0,0] 3...5 ~p1}
\end{center}
\end{example}

\end{document}
