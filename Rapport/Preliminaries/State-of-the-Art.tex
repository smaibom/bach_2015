\section{State-of-the-Art}
The current tools readily available that provides scan\_for\_matches like 
functionality are RE2, Google's regular expression library because of 
how it handles alternations as well as its linear running time and TRE, 
python's regex library and non-deterministic grep (NR-grep), since all three 
allows errors in its results and supports backreferencing. %TODO: CHECK BACKREF
 \subsection{RE2} %Google's RE engine
 RE2 is Google's regular expression library written in C++. It does not 
 support backreferencing, but claims to run faster if a pattern has a high degree 
 of alternations~\cite{javare2}. Due to RE2's lack of backreferencing and support for 
 matching with errors, it would be unable to properly reproduce 
 scan\_for\_matches' functionality. The project can be found on its github 
 page: \url{https://github.com/google/re2}.
 \subsection{TRE} %Got both backtracking and backreferencing
 TRE was created by Ville Laurikari for his master's thesis in 
 2001\cite{LaurikariComplex}, and is a regular expression engine which supports 
 backreferences and matching with errors. Because of this, TRE is the best 
 candidate for modification in order to simulate scan\_for\_matches 
 functionality (see Section~\ref{tre}).
 \subsection{Python's Regex Library} %Group ref. instead of backref
 The python module regex~\cite{pythonregex} is an alternative regular expression 
 module to the native python module re created by Matt. It allows the 
 specification of mismatches in its search terms, however because of the lack 
 of theoretical documentation pertaining to the module and the nature of 
 the language the module was written for, we didn't see the regex library as 
 a potential alternative for scan\_for\_matches. 
 \subsection{NR-grep}
 NR-grep~\cite{Navarro00nr-grep:a} is a pattern matching tool written 
 in C by Gonzalo Navarro in 2000. It allows backreferences as well as matching 
 with errors, and would have been a candidate for modification alongside TRE had 
 we learned about it earlier. However, because we learned about the tool 
 relatively late in the process, we did not have time to work with it.
