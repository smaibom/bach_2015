\section{Scan\_For\_Matches}\label{scanformatches}
Scan\_for\_matches is a pattern-matching tool created by Ross Overbeek, David 
Joerg and Morgan Price in C which searches through data files\footnote{http://blog.theseed.org/servers/2010/07/scan-for-matches.html}. Users specify 
what they want to search for by defining a pattern, and scan\_for\_matches 
returns all matches that corresponds to the specified pattern. 
\begin{mydef}\label{patex}
Let $\Sigma$ denote an alphabet. Then we can define a \emph{pattern unit} as follows:\\
\begin{tabular}{|r|l|}
\hline
\textbf{Pattern Unit}&\textbf{Function of Pattern Unit}\\
\hline
{\tt h}&Match the sequence h, where $h\in\Sigma*$\\
\hline
{\tt n...m}&Match n to m characters where 0 $\leq$ n $\leq$ m\\
\hline
{\tt x=n...m}&Match n to m characters, and label the sequence x\\
\hline
{\tt x $|$ y}&Match either pattern x or pattern y\\
\hline
{\tt x[n,m,l]}&\pbox{20cm}{Match pattern x, allowing for n mismatches, m deletions and l insertions\\ where n,m,l $\geq$ 0}\\
\hline
{\tt length(x+y) $<$ n}&The length of patterns x+y$<$n where n $>$ 0\\
\hline
{\tt z=\{uv, vu\}}&\pbox{20cm}{Create a pattern rule where u is the complement of v, and v is the\\ complement of u,
                               where $u,~v\in\Sigma$, and call the rule z}\\
\hline
{\tt $<$x}&Match the reverse of pattern x\\
\hline
{\tt $\sim$x}&\pbox{20cm}{Match the reverse complement of pattern x using the G-C, \\C-G, A-T and T-A pairing rule}\\
\hline
{\tt z$\sim$x}&\pbox{20cm}{Match the reverse complement of pattern x using pattern rule z=\{uv,vu\}}\\
\hline
{\tt \textasciicircum ~x}&\pbox{20cm}{Match only pattern x if it is at the start of a string}\\
\hline
{\tt x \$}&Match only pattern x if it is at the end of a string\\
\hline
\end{tabular}
\end{mydef}

\begin{mydef}\label{patc}
Let $\Lambda$ be any pattern unit in definition~\ref{patex}. Let ${\tt E}\in\Lambda$.
Let {\tt 0} be the empty string. Let $P$ be a pattern that we are processing.
A pattern may then be constructed as such: 
\begin{align*}
{\tt P~}~=&~{\tt E~P~|~0}
\end{align*}
\end{mydef}
Definition~\ref{patc} states that a pattern may be any combination of the pattern 
units defined in definition~\ref{patex}.
\begin{mydef}\label{patlint}
Let $\Sigma$ be an alphabet. Let {\tt a $\in\Sigma$}. Let $\epsilon$ be the empty 
string.
Then the language interpretation of definition~\ref{patex} is defined as follows:
\begin{align*}
L({\tt \epsilon})~&=~\emptyset\\
L({\tt a})~&=~\{{\tt a}\}\\
L({\tt E_1~E_2})~&=L({\tt E_1})~L({\tt E_2})\\
L({\tt E_1~|~E_2})~&=~L({\tt E}_{\tt 1})~\cup~L({\tt E}_{\tt 2})\\
L({\tt E})^0~&=~L({\tt \epsilon})\\
L({\tt E})^n~&=~\underbrace{L({\tt E})L({\tt E})...L({\tt E})}_\text{n}\\
L({\tt n...n})~&=~L({\tt E})^n\\
L({\tt n...m})~&=~L({\tt n...n})\cup L({\tt n+1...n+1})\cup...\cup L({\tt m-1...m-1})\cup L({\tt m...m})~=~\bigcup\limits_{n={\tt n}}^{\tt m} L(n...n)\\
f(L(\Sigma))~&=~\bigcup\limits_{x\in\Sigma} f(L(x))\\ %f(E)
L({\tt <E})~&=~\{w^R~|~w\in E\}\\
L({\tt {\sim}E})~&=~f(L(<\Sigma))\\
L({\tt length(E<n)} )~&=~\{{\tt |v|<n~|~v\in E},~0~{\tt <n}\}
\end{align*}
\end{mydef}
Definition~\ref{patlint} defines the language interpretation of 
scan\_for\_matches. $f(L(\Sigma))$ defines a mapping \cite[p. 60]{Hopcroft1979} 
which substitutes the current character with the complementary in alphabet 
$\Sigma$.
%Needs something more here
Below is an example of a scan\_for\_matches pattern.
\begin{myex}\label{stemex}
Say we want to write a pattern that finds the sequence {\tt GUUC}, allowing 
one mismatch, followed by a random sequence which has a length between 3 and 5, 
followed by the reverse complement of the first sequence that we found. We can 
then write this as \begin{center}
{\tt p1=GUUC[1,0,0] 3...5 \textasciitilde p1}
\end{center}
\end{myex}
Example \ref{stemex} matches a stem loop as described in section 
\ref{structs}. Note that if we wanted to find all stem loops in a file where 
the bonded bases are of length 
4, we would replace {\tt GUUC[1,0,0]} with an arbitrary sequence of characters of length $4$ 
by writing {\tt p1=4..4 3..5 \textasciitilde p1}.
