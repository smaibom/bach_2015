
\section{Scan\_For\_Matches}\label{scanformatches}
Scan\_for\_matches is a string-searching tool created by Ross Overbeek, David 
Joerg and Morgan Price in C which searches through text files. Users specify 
what they wish to search for by defining a pattern, and scan\_for\_matches 
returns all matches that corresponds to the specified pattern. 
\begin{mydef}\label{patc}
Let {\tt E} be any pattern that's in the alphabet $\Sigma$ as defined in Example \ref{patex}. 
Let 0 be the empty string.
Let {\tt A} be a string that we are processing to see if the pattern is valid.
A pattern may then be constructed as such: 
\end{mydef}

\begin{myex}\label{patex}
Let $\Sigma$ denote an alphabet. Then we can define a pattern as follows:\\
\begin{tabular}{|r|l|}
\hline
{\tt h}&Match the sequence h, where $h\in\Sigma$\\
\hline
{\tt n...m}&Match n to m characters where n $\leq$ m\\
\hline
{\tt x=n...m}&Match n to m characters, and call the sequence x\\
\hline
{\tt x $|$ y}&Match either x or y\\
\hline
{\tt x[n,m,l]}&Match x, allowing for n mismatches, m deletions and insertions where n,m,l $\geq$ 0\\
\hline
{\tt length(x+y) $<$ n}&The length of patterns x+y$<$n where n $>$ 0\\
\hline
{\tt z=\{hl, lh\}}&\pbox{20cm}{Create a pattern rule where h is the complement of l, and l is the complement\\ of h,
                               where $h\in\Sigma,~l\in\Sigma$, and call it z}\\
\hline
{\tt $<$x}&Match the reverse of pattern x\\
\hline
{\tt \textasciitilde x}&\pbox{20cm}{Match the reverse complement of pattern x using the G-C, \\C-G, A-T and T-A pairing rule}\\
\hline
{\tt z\textasciitilde x}&\pbox{20cm}{Match the reverse complement of pattern x using pattern rule z}\\
\hline
{\tt \textasciicircum ~x}&\pbox{20cm}{Match only pattern x if it is at the start of a string}\\
\hline
{\tt x \$}&Match only pattern x if it is at the end of a string\\
\hline
\end{tabular}
\end{myex}

Definition \ref{patc} states that a pattern may be any combination of the alphabet 
defined in example \ref{patex}.
Using these patterns, it is possible to make very specific or very broad 
searches in a text file. 

\begin{myex}\label{stemex}
Say we want to write a pattern that finds the sequence {\tt GUUC}, allowing 
one mismatch, followed by a random sequence which has a length between 3 and 5, 
followed by the reverse complement of the first sequence that we found. We can 
then write this as \begin{center}
{\tt p1=GUUC[1,0,0] 3...5 \textasciitilde p1}
\end{center}
\end{myex}
Example \ref{stemex} will match a stem loop as described in section 
\ref{structs}. Note that if we wanted to find all stem loops in a file where 
the bonded bases are of length 
4, we would replace {\tt GUUC[1,0,0]} with an arbitrary sequence of characters 
by writing {\tt p1=4..4 3..5 \textasciitilde p1}. 
