\subsection{Tagged NFA}
Before introducing tagged NFA, we must define what a missmatch is. 
\begin{mydef}
A missmatch $M$ in a string $n$ with a alphabet $\Sigma$ is one of 3 types, insertion, deletion and alteration. An insertion is where a symbol in $\Sigma$ is added to $n$. A deletion is where a symbol in $n$ is removed. A alteration is where a symbol in $n$ is changed with a symbol in $\Sigma$. 
\end{mydef}
NFA's do not handle missmatching in strings per default, to introduce this, a new set of transitions is added to the NFA. This type of NFA is called a Tagged NFA. 
\begin{mydef}
A tagged NFA is a 6 tuple $(Q,\Sigma,\Delta,q^s,q^a,\Delta')$ where the first 5 elements is a standard NFA and $\Delta'$ is a set of 4 tuples containing $\epsilon$ transitions for missmatches. The type of $\Delta'$ is $\Delta' \subseteq Q ~x~ \{\epsilon\} ~x~ Q ~x~ M$.  
\end{mydef}

\subsection{Constructing TNFA}
TNFAs are constructed the same way as shown in tabel \ref{tab:NFA_TAB}, with the exception when constructing literals. A new set of $\epsilon$-transitions is added as shown in tabel \ref{tab:TNFA} as the new literal construction rule. The new transitions added in $\Delta'$ is shown as a red arrow which denotes a deletion transition, a green arrow for the insertion transition and a blue arrow for the alteration transition.
\begin{table}[h]\label{nfac}
\caption{Translating table for literal construction of TNFA}
\centering
\begin{tabular}{*{2}{m{0.4\textwidth}}}
\hline
\begin{center}$a$\end{center} &\begin{center}
\begin{tikzpicture}[->,>=stealth,shorten >=1pt,auto,node distance=2 cm, scale = 0.75, transform shape,initial text={}]
  \node [initial, state] (0) {};
  \node [accepting,state, right of=0] (1) {};

  \path[->] (0) edge node [above] {$a$} (1);
  \path[->] (0) edge [color=green, in=100,out=80,loop] node [color=black, above] {$\epsilon$} (0);
  \path[->] (0) edge [color=red,bend left] node [color=black, above] {$\epsilon$} (1);

  \path[->] (0) edge [color=blue,bend right] node [color=black, below] {$\epsilon$} (1);
\end{tikzpicture}\end{center}\\
\hline
\end{tabular}
\label{tab:TNFA}
\end{table}


