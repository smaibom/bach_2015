\section{Tagged NFA}
\label{sec:tnfa}
The tagged NFA(TNFA) is introduced in \cite{Ville}. It introduces the concept of tagging NFA transitions. A TNFA is a NFA which allows for mismatches in the input string.

\begin{mydef}
A tagged NFA is a 6 tuple $(Q,\Sigma,\Delta,q^s,q^a,\Delta')$ where the first 5 elements is a standard NFA and $\Delta'$ is a set of tuples containing transitions for mismatches. The type of $\Delta'$ is $\Delta' \subseteq Q \cdot\{\epsilon\} \cdot Q \cdot M$, where $M = \{i,d,a\}$ for mismatch types.  
\end{mydef}
TNFA adds a new l1iteral construction rule. A new set of tagged $\epsilon$-transitions is added as shown in Tabel \ref{tab:TNFA} on the new literal construction rule. Transitions in $\Delta'$ are represented as $(q,\epsilon,q,i) \in \Delta'$ for insertions,$(q,\epsilon,q',d) \in \Delta'$ for deletions and $(q,\epsilon,q',a) \in \Delta'$ for alterations, we write these transitions as $(q {\color{green} \xrightarrow{{\color{black}\epsilon/i}}} q)$ for insertions, $(q {\color{red} \xrightarrow{{\color{black}\epsilon/d}}} q')$ for deletions and $(q {\color{blue} \xrightarrow{{\color{black}\epsilon/a}}} q')$ for alterations.
\begin{table}[h]\label{nfac}
\caption{Translating table for literal construction of TNFA}
\label{nfac}
\centering
\begin{tabular}{*{2}{m{0.4\textwidth}}}
\hline
\begin{center}$0$\end{center} &\begin{center}
\begin{tikzpicture}[->,>=stealth,shorten >=1pt,auto,node distance=2 cm, scale = 0.75, transform shape,initial text={}]
  \node [initial, state] (0) {};
  \node [accepting,state, right of=0] (1) {};

\end{tikzpicture}\end{center} \\
\hline
\begin{center}$1$\end{center} &\begin{center} 
\begin{tikzpicture}[->,>=stealth,shorten >=1pt,auto,node distance=2 cm, scale = 0.75, transform shape,initial text={}]
  \node [initial, state] (0) {};
  \node [accepting,state, right of=0] (1) {};

  \path[->] (0) edge node [above] {$\epsilon$} (1);

\end{tikzpicture} \\
\end{center} \\
\hline
\begin{center}$a$\end{center} &\begin{center}
\begin{tikzpicture}[->,>=stealth,shorten >=1pt,auto,node distance=2 cm, scale = 0.75, transform shape,initial text={}]
  \node [initial, state] (0) {};
  \node [accepting,state, right of=0] (1) {};

  \path[->] (0) edge node [above] {$a$} (1);
  \path[->] (0) edge [color=green, in=100,out=80,loop] node [color=black, above] {$\epsilon/i$} (0);
  \path[->] (0) edge [color=red,bend left] node [color=black, above] {$\epsilon/d$} (1);

  \path[->] (0) edge [color=blue,bend right] node [color=black, below] {$\epsilon/a$} (1);
\end{tikzpicture}\end{center}\\
\hline
\begin{center}$E^1E^2$\end{center} &\begin{center} 
\begin{tikzpicture}[->,>=stealth,shorten >=1pt,auto,node distance=1.5 cm, scale = 0.75, transform shape,initial text={}]
  \node [initial, state] (0) {};
  \node [right of= 0] (1) {$E^1$};
  \node [state,right of= 1] (2) {};
  \node [right of= 2] (3) {$E^2$};
  \node [accepting, state,right of= 3] (4) {};

  \path[-] (0) edge node {} (1);
  \path[-] (2) edge node {} (3);

  \path[->] (1) edge node {} (2);
  \path[->] (3) edge node {} (4);

\node [draw=red, fit= (0) (2),dashed] {};
\node [draw=blue, fit= (2) (4),dashed,inner sep=0.25cm] {};
  
\end{tikzpicture}\end{center} \\
\hline
\begin{center}$E^1 + E^2$\end{center} &\begin{center} 
\begin{tikzpicture}[->,>=stealth,shorten >=1pt,auto,node distance=1.5 cm, scale = 0.75, transform shape,initial text={}]
  \node [initial, state] (0) {};
  \node [state,above right of=0,xshift=1cm] (1) {};
  \node [state,below right of=0,xshift=1cm] (2) {};
  \node [right of =1] (3) {$E^1$};
  \node [right of=2] (4) {$E^2$};
  \node [state, right of=3] (5) {};
  \node [state, right of=4] (6) {};
  \node [accepting,state,below right of=5,xshift=1cm] (7) {};

  \path[-] (1)edge node {} (3);
  \path[-] (2)edge node {} (4);

  \path[->] (0) edge node {$\epsilon$} (1);
  \path[->] (0) edge node[below] {$\epsilon$} (2);
  \path[->] (3) edge node {} (5);
  \path[->] (4) edge node {} (6);
  \path[->] (5) edge node {$\epsilon$} (7);
  \path[->] (6) edge node[below] {$\epsilon$} (7);


\node [draw=red, fit= (1) (5),dashed] {};
\node [draw=blue, fit= (2) (6),dashed] {};

\end{tikzpicture} \end{center} \\
\hline
\begin{center}$E^*$\end{center} &\begin{center} 
\begin{tikzpicture}[->,>=stealth,shorten >=1pt,auto,node distance=1.5 cm, scale = 0.75, transform shape,initial text={}]
  \node [initial, state] (0) {};
  \node [accepting,state,right of=0,xshift=2cm] (1) {};
  \node [state,above left of=0,xshift=-1cm,yshift=1cm] (2) {};
  \node [state,above right of=0,xshift=1cm,yshift=1cm] (3) {};
  \node [right of=2,xshift=0.60cm] (4) {$E$};

  \path[->] (0) edge node {$\epsilon$} (1);
  \path[->] (3) edge node {$\epsilon$} (0);
  \path[->] (0) edge node {$\epsilon$} (2);
  \path[->] (4) edge node {} (3);

  \path[-] (2) edge node {} (4);

\node [draw=red, fit= (2) (3),dashed] {};
\end{tikzpicture}
\end{center} \\
\hline
\end{tabular}
\label{tab:TNFA}
\end{table}