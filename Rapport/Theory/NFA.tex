\section{Nondeterministic Finite Automaton}\label{RA_TO_NFA}
A nondeterministic finite automaton(NFA) can be used to determine if a input string is matches a particular set of strings.
\begin{mydef}
An nondeterministic finite automaton (NFA) is a 5-tuple $(Q,\Sigma,\Delta,q^s ,q^a)$, where $Q$ is a finite set of states, $\Sigma$ is the input alphabet, the initial state $q^s \in Q$,  the accepting state $q^a \in Q$ and $\Delta \subseteq Q \cdot (\Sigma \cup \{\epsilon\}) \cdot Q$ is the transition relation. 
\end{mydef}
Each RE can be converted to an NFA, and vice versa. Table~\ref{tab:NFA_TAB} shows how each RE can be converted into an NFA. The states in $Q$ are represented as circles. The initial state $q^s$ is shown by adding a small arrow pointing to it, and the accepting state $q^a$ is shown as a double circle. Transitions in $\Delta$ are represented as  $(q,a,q')\in \Delta$ or $(q,\epsilon ,q')\in\Delta$, we write these transitions as $q \xrightarrow{a} q'$ or $q \xrightarrow{\epsilon} q'$. For subexpressions we use boxes and split the transition arrow, marking it with an $E$, to denote the NFA resulted from converting the expression. 
%\\
%(( NOTE: We'll include a table with conversions ))
%\\
%With little effort every regular expression can be translated into a graph, which can then be analysed.

\begin{table}[h]
\caption{Translation table from regular expressions to NFA}
\centering
\begin{tabular}{*{2}{m{0.4\textwidth}}}
\hline
\begin{center}$0$\end{center} &\begin{center}
\begin{tikzpicture}[->,>=stealth,shorten >=1pt,auto,node distance=2 cm, scale = 0.75, transform shape,initial text={}]
  \node [initial, state] (0) {};
  \node [accepting,state, right of=0] (1) {};

\end{tikzpicture}\end{center} \\
\hline
\begin{center}$1$\end{center} &\begin{center} 
\begin{tikzpicture}[->,>=stealth,shorten >=1pt,auto,node distance=2 cm, scale = 0.75, transform shape,initial text={}]
  \node [initial, state] (0) {};
  \node [accepting,state, right of=0] (1) {};

  \path[->] (0) edge node [above] {$\epsilon$} (1);

\end{tikzpicture} \\
\end{center} \\
\hline
\begin{center}$a$\end{center} &\begin{center} 
\begin{tikzpicture}[->,>=stealth,shorten >=1pt,auto,node distance=2 cm, scale = 0.75, transform shape,initial text={}]
  \node [initial, state] (0) {};
  \node [accepting,state, right of=0] (1) {};

  \path[->] (0) edge node [above] {$a$} (1);

\end{tikzpicture}\end{center} \\
\hline
\begin{center}$E^1E^2$\end{center} &\begin{center} 
\begin{tikzpicture}[->,>=stealth,shorten >=1pt,auto,node distance=1.5 cm, scale = 0.75, transform shape,initial text={}]
  \node [initial, state] (0) {};
  \node [right of= 0] (1) {$E^1$};
  \node [state,right of= 1] (2) {};
  \node [right of= 2] (3) {$E^2$};
  \node [accepting, state,right of= 3] (4) {};

  \path[-] (0) edge node {} (1);
  \path[-] (2) edge node {} (3);

  \path[->] (1) edge node {} (2);
  \path[->] (3) edge node {} (4);

\node [draw=red, fit= (0) (2),dashed] {};
\node [draw=blue, fit= (2) (4),dashed,inner sep=0.25cm] {};
  
\end{tikzpicture}\end{center} \\
\hline
\begin{center}$E^1 + E^2$\end{center} &\begin{center} 
\begin{tikzpicture}[->,>=stealth,shorten >=1pt,auto,node distance=1.5 cm, scale = 0.75, transform shape,initial text={}]
  \node [initial, state] (0) {};
  \node [state,above right of=0,xshift=1cm] (1) {};
  \node [state,below right of=0,xshift=1cm] (2) {};
  \node [right of =1] (3) {$E^1$};
  \node [right of=2] (4) {$E^2$};
  \node [state, right of=3] (5) {};
  \node [state, right of=4] (6) {};
  \node [accepting,state,below right of=5,xshift=1cm] (7) {};

  \path[-] (1)edge node {} (3);
  \path[-] (2)edge node {} (4);

  \path[->] (0) edge node {$\epsilon$} (1);
  \path[->] (0) edge node[below] {$\epsilon$} (2);
  \path[->] (3) edge node {} (5);
  \path[->] (4) edge node {} (6);
  \path[->] (5) edge node {$\epsilon$} (7);
  \path[->] (6) edge node[below] {$\epsilon$} (7);


\node [draw=red, fit= (1) (5),dashed] {};
\node [draw=blue, fit= (2) (6),dashed] {};

\end{tikzpicture} \end{center} \\
\hline
\begin{center}$E^*$\end{center} &\begin{center} 
\begin{tikzpicture}[->,>=stealth,shorten >=1pt,auto,node distance=1.5 cm, scale = 0.75, transform shape,initial text={}]
  \node [initial, state] (0) {};
  \node [accepting,state,right of=0,xshift=2cm] (1) {};
  \node [state,above left of=0,xshift=-1cm,yshift=1cm] (2) {};
  \node [state,above right of=0,xshift=1cm,yshift=1cm] (3) {};
  \node [right of=2,xshift=0.60cm] (4) {$E$};

  \path[->] (0) edge node {$\epsilon$} (1);
  \path[->] (3) edge node {$\epsilon$} (0);
  \path[->] (0) edge node {$\epsilon$} (2);
  \path[->] (4) edge node {} (3);
  \path[-] (2) edge node {} (4);

\node [draw=red, fit= (2) (3),dashed] {};

\end{tikzpicture}
\end{center} \\
\hline
\end{tabular}
\label{tab:NFA_TAB}
\end{table}

\begin{myex}
In example \ref{reinter}, the expression ${\tt c}({\tt a}+{\tt b})^*$ was used. The NFA buildup is done like the language interpretation in the example. We first construct the $E_1E_2$ expression, where $E_1 = {\tt c}$ and $E_2 = ({\tt a}+{\tt b})^*$.

\begin{center} 
\begin{tikzpicture}[->,>=stealth,shorten >=1pt,auto,node distance=1.5 cm, scale = 0.75, transform shape,initial text={}]
  \node [initial, state] (0) {};
  \node [right of= 0] (1) {$E^1$};
  \node [state,right of= 1] (2) {};
  \node [right of= 2] (3) {$E^2$ };
  \node [accepting, state,right of= 3] (4) {};

  \path[-] (0) edge node {} (1);
  \path[-] (2) edge node {} (3);

  \path[->] (1) edge node {} (2);
  \path[->] (3) edge node {} (4);
  
\node [draw=red, fit= (0) (2),dashed] {};
\node [draw=blue, fit= (2) (4),dashed,inner sep=0.25cm] {};
\end{tikzpicture}\\
\end{center} 
We first construct $E_1$
\begin{center}
\begin{tikzpicture}[->,>=stealth,shorten >=1pt,auto,node distance=1.5 cm, scale = 0.75, transform shape,initial text={}]
  \node [initial, state] (0) {};
  \node [state,right of= 0,xshift=1.5cm] (2) {};
  \node [right of= 2] (3) {$E^2$ };
  \node [accepting, state,right of= 3] (4) {};

  \path[-] (2) edge node {} (3);

  \path[->] (0) edge node {{\tt c}} (2);
  \path[->] (3) edge node {} (4);
  
\node [draw=blue, fit= (2) (4),dashed,inner sep=0.25cm] {};
\end{tikzpicture}\\
\end{center} 
Constructing $E_2$ results in a new subexpression $E_3 = {\tt a}+{\tt b}$.
\begin{center}
\begin{tikzpicture}[->,>=stealth,shorten >=1pt,auto,node distance=1.5 cm, scale = 0.75, transform shape,initial text={}]
  \node [initial,state] (99) {};
  \node [state,right of= 99,xshift=2cm] (0) {};
  \node [accepting,state,right of=0,xshift=2cm] (1) {};
  \node [state,above left of=0,xshift=-1cm,yshift=1cm] (2) {};
  \node [state,above right of=0,xshift=1cm,yshift=1cm] (3) {};
  \node [right of=2,xshift=0.60cm] (4) {$E_3$};

  \path[->] (0) edge node {$\epsilon$} (1);
  \path[->] (3) edge node {$\epsilon$} (0);
  \path[->] (0) edge node {$\epsilon$} (2);
  \path[->] (4) edge node {} (3);
  \path[->] (99) edge node {{\tt c}} (0);
  \path[-] (2) edge node {} (4);
\node [draw=red, fit= (2) (3),dashed] {};
\end{tikzpicture}\\
\end{center}
$E_3$ produces the two new expressions $E_4 ={\tt a}$ and $E_5 ={\tt b}$. 

\begin{center}
\begin{tikzpicture}[->,>=stealth,shorten >=1pt,auto,node distance=1.5 cm, scale = 0.75, transform shape,initial text={}]
  \node [initial,state] (99) {};
  \node [state,right of= 99,xshift=1.5cm] (0) {};
  \node [accepting,state,right of=0,xshift=2.5cm] (1) {};
  
  \node [state,above left of=0,xshift=-2.5cm,yshift=2cm] (2) {};
  \node [state,above right of=2,xshift=1.25cm] (6) {};
  \node [right of=6] (61) {$E_4$};
  \node [state,right of=61] (8) {};
  
  \node [state,above right of=0,xshift=3cm,yshift=2cm] (3) {};
  \node [state,below right of=2,xshift=1.25cm] (5) {};
  \node [right of=5] (51) {$E_5$};
  \node [state,right of=51] (7) {};

  \path[->] (2) edge node {$\epsilon$} (5)
          edge node {$\epsilon$} (6)
        (7) edge node {$\epsilon$} (3)
        (8) edge node {$\epsilon$} (3);

  \path[-] (6) edge node {} (61)
         (5) edge node {} (51);

  \path[->] (0) edge node {$\epsilon$} (1)
        (61) edge node {} (8)
        (51) edge node {} (7);
  \path[->] (3) edge [bend left,above] node {$\epsilon$} (0);
  \path[->] (0) edge [bend left,above] node {$\epsilon$} (2);
  \path[->] (99) edge node {{\tt c}} (0);


\node [draw=red, fit= (6) (8),dashed] {};
\node [draw=blue, fit= (5) (7),dashed] {};
\end{tikzpicture}
\end{center}
After constructing $E_4$ 
\begin{center}
\begin{tikzpicture}[->,>=stealth,shorten >=1pt,auto,node distance=1.5 cm, scale = 0.75, transform shape,initial text={}]
  \node [initial,state] (99) {};
  \node [state,right of= 99,xshift=1.5cm] (0) {};
  \node [accepting,state,right of=0,xshift=2.5cm] (1) {};
  
  \node [state,above left of=0,xshift=-2.5cm,yshift=2cm] (2) {};
  \node [state,above right of=2,xshift=1.25cm] (6) {};
  \node [state,right of=6,xshift=1.5cm] (8) {};
  
  \node [state,above right of=0,xshift=3cm,yshift=2cm] (3) {};
  \node [state,below right of=2,xshift=1.25cm] (5) {};
  \node [right of=5] (51) {$E_5$};
  \node [state,right of=51] (7) {};

  \path[->] (2) edge node {$\epsilon$} (5)
          edge node {$\epsilon$} (6)
        (7) edge node {$\epsilon$} (3)
        (8) edge node {$\epsilon$} (3);

  \path[-] (5) edge node {} (51);

  \path[->] (0) edge node {$\epsilon$} (1)
        (6) edge node {{\tt a}} (8)
        (51) edge node {} (7);
  \path[->] (3) edge [bend left,above] node {$\epsilon$} (0);
  \path[->] (0) edge [bend left,above] node {$\epsilon$} (2);
  \path[->] (99) edge node {{\tt c}} (0);


\node [draw=blue, fit= (5) (7),dashed] {};
\end{tikzpicture}
\end{center}
After constructing $E_5$ we end up with the NFA of the expression $E$
\begin{center}
\begin{tikzpicture}[->,>=stealth,shorten >=1pt,auto,node distance=1.5 cm, scale = 0.75, transform shape,initial text={}]
  \node [initial,state] (99) {};
  \node [state,right of= 99,xshift=1.5cm] (0) {};
  \node [accepting,state,right of=0,xshift=2.5cm] (1) {};
  
  \node [state,above left of=0,xshift=-2.5cm,yshift=2cm] (2) {};
  \node [state,above right of=2,xshift=1.25cm] (6) {};
  \node [state,right of=6,xshift=1.5cm] (8) {};
  
  \node [state,above right of=0,xshift=3cm,yshift=2cm] (3) {};
  \node [state,below right of=2,xshift=1.25cm] (5) {};
  \node [state,right of=5,xshift=1.5cm] (7) {};

  \path[->] (2) edge node {$\epsilon$} (5)
          edge node {$\epsilon$} (6)
        (7) edge node {$\epsilon$} (3)
        (8) edge node {$\epsilon$} (3);

  \path[->] (0) edge node {$\epsilon$} (1)
        (6) edge node {{\tt a}} (8)
        (5) edge node {{\tt b}} (7);
  \path[->] (3) edge [bend left,above] node {$\epsilon$} (0);
  \path[->] (0) edge [bend left,above] node {$\epsilon$} (2);
  \path[->] (99) edge node {{\tt c}} (0);

\end{tikzpicture}
\end{center}
\end{myex}