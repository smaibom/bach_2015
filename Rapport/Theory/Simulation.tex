\subsection{Matching using NFA}
To match if a given string is accepted in an NFA, two functions $\epsilon$-$closure$ and $reachable$ of the simulation Algorithm \ref{nfaalg1} are introduced.
\begin{mydef}
Given a set of NFA states M, the $\epsilon $-$ closure$ of M is a set of states that are reachable from states in M by following any number of $\epsilon$-transitions in $\Delta$.
\begin{center}
$\epsilon$-$closure(M)$ =$ M \cup \{q'|q\in \epsilon$-$closure(M) ~and ~(q,\epsilon,q') ~ \in ~ \Delta\}$
\end{center}
\cite[p. 34, def 2.2]{compile}
\end{mydef}

\begin{mydef}
Given a set of NFA states and an input symbol {\tt a}, the $reachable$ states of $M$ are a set of states that are reachable from states in M by following transitions in $\Delta$ which match the input symbol a. 
\begin{center}
$reachable(M,a)$ = $\{q'|q \in M,(q,a,q')\in \Delta \}$
\end{center}
\end{mydef}
\begin{algorithm}
  \caption{NFA simulation}
    \label{nfaalg1}
  \begin{algorithmic}[1]
    \Require{$N$ is an NFA and $s$ is a string}
    \Ensure{True if $s$ is accepted in $N$, False if $s$ is rejected}
    \Function{Simulation}{$N(Q,\Sigma,\Delta,q^s,q^a), s$}
      \Let{$stateset$}{$\{q^s\}$} 
      \For{{\bf each} symbol {\bf in} $s$}
        \If{$stateset = \emptyset$} 
            \State \Return{False}
        \EndIf
        \Let{$next$}{$\emptyset$}
        \Let{$states$}{$\epsilon$-closure($stateset$)}
        \Let{$next$}{$reachable(states,symbol)$}
        \Let{$stateset$}{$next$}
      \EndFor
      \If{$q^a \in stateset$}
        \State \Return{True}
      \EndIf
      \State \Return{False}
    \EndFunction
  \end{algorithmic}
\end{algorithm}

\begin{myex}\label{simsuc}
Given the RE  $E={\tt c}({\tt ab}+{\tt a})^*{\tt b}$, the resulting NFA $N$ is seen in Figure \ref{nfasimsucc}

\begin{figure}[h!]
\begin{center}
\begin{tikzpicture}[->,>=stealth,shorten >=1pt,auto,node distance=1.5 cm, scale = 0.75, transform shape,initial text={}]
  \node [initial,state] (99) {0};
  \node [state,right of= 99,xshift=2cm] (0) {1};
  \node [accepting,state,right of=0,xshift=2cm] (1) {9};
  \node [state,above left of=0,xshift=-2.5cm,yshift=2cm] (2) {2};
  \node [state,above right of=0,xshift=5.5cm,yshift=2cm] (3) {8};
  \node [state,below right of=2,xshift=1.25cm] (5) {6};
  \node [state,above right of=2,xshift=1.25cm] (6) {3};
  \node [state,right of=5,xshift=1.25cm] (7) {7};
  \node [state,right of=6,xshift=1.25cm] (8) {4};

  \node [state,right of=8,xshift=1.25cm] (97) {5};
  \path[->] (2) edge node {$\epsilon$} (5)
          edge node {$\epsilon$} (6)
        (5) edge node {{\tt a}} (7)
        (6) edge node {{\tt a}} (8)
        (7) edge node {$\epsilon$} (3)
        (8) edge node {{\tt b}} (97)
        (97) edge node {$\epsilon$} (3);

  \path[->] (0) edge node {{\tt b}} (1);
  \path[->] (3) edge node {$\epsilon$} (0);
  \path[->] (0) edge node {$\epsilon$} (2);
  \path[->] (99) edge node {{\tt c}} (0);
  \end{tikzpicture}
  \end{center}
  \caption{Figure of the NFA from the $E$ = {\tt c}({\tt ab}+{\tt a})$^*${\tt b}}
  \label{nfasimsucc}
\end{figure}
We now want to see if the input string {\tt caaabb} is accepted into N. The initial stateset \{$q^s\}=\{0\}$ \\
\begin{tabular}{l l l}
Simulate(N,{\tt caaabb}) & & \\
symbol {\tt c}: & $\epsilon$-$closure(\{0\})$ &$ = \{0\}$\\
&$reachable(\{0\},{\tt c})$& = $\{1\}$\\
symbol {\tt a}: & $\epsilon$-$closure(\{1\})$& = $\{1,2,3,6\}$\\
&$reachable(\{1,2,3,6\},{\tt a})$ & = $\{4,7\}$\\
symbol {\tt a}: & $\epsilon$-$closure(\{4,7\})$& = $\{1,2,3,4,6,7,8\}$\\
&$reachable(\{1,2,3,4,6,7,8\},{\tt a})$ &= $\{4,7\}$\\
symbol {\tt a}: & $\epsilon$-$closure(\{4,7\})$& = $\{1,2,3,4,6,7,8\}$\\
&$reachable(\{1,2,3,4,6,7,8\},{\tt a})$ &= $\{4,7\}$\\
symbol {\tt b}: & $\epsilon$-$closure(\{4,7\})$& = $\{1,2,3,4,6,7,8\}$\\
&$reachable(\{1,2,3,4,6,7,8\},{\tt b})$ &= $\{5,9\}$\\
symbol {\tt b}: & $\epsilon$-$closure(\{5,9\})$& = $\{1,2,3,5,6,8,9\}$\\
&$reachable(\{1,2,3,6,5,8,9\},{\tt b})$ &= $\{9\}$\\
\end{tabular}\\
After the final input symbol, it can be seen that the accepting state $q^a$ is in the final stateset. So the string {\tt caaabb} is accepted in N.
\end{myex}
\begin{myex}
Using the NFA N from Example \ref{simsuc} where $q^s=0$, the simulation attempts to check if string {\tt cabbbb} is accepted in N.\\
\begin{tabular}{l l l}
Simulate(N,{\tt cabbbb}) & & \\
symbol {\tt c}: & $\epsilon$-$closure(\{0\})$ &$ = \{0\}$\\
&$reachable(\{0\},{\tt c})$& = $\{1\}$\\
symbol {\tt a}: & $\epsilon$-$closure(\{1\})$& = $\{1,2,3,6\}$\\
&$reachable(\{1,2,3,6\},{\tt a})$ & = $\{4,7\}$\\
symbol {\tt b}: & $\epsilon$-$closure(\{4,7\})$& = $\{1,2,3,4,6,7,8\}$\\
&$reachable(\{1,2,3,4,6,7,8\},{\tt b})$ &= $\{5,9\}$\\
symbol {\tt b}: & $\epsilon$-$closure(\{5,9\})$& = $\{1,2,3,5,6,8,9\}$\\
&$reachable(\{1,2,3,6,5,8,9\},{\tt b})$ &= $\{9\}$\\
symbol {\tt b}: & $\epsilon$-$closure(\{9\})$& = $\{9\}$\\
&$reachable(\{9\},{\tt b})$ &= $\emptyset$\\
\end{tabular}\\
The $\emptyset$ is reached at the 5'th input symbol of {\tt cabbbb}, resulting in the simulation to fail. 
\end{myex}