\newpage
\subsection{Simulating TNFA}\label{section_TNFA}
TNFA simulation adds an additional argument for amount of mismatches allowed. The stateset contains 4 tuples of $(Q,i,d,a)$ where i,d and a is a count of how many of each transition has been used. The $\epsilon-closure$ and $reachable$ functions are extended to allow for for a set of 4 tuples.
\begin{mydef} 
Given a 4 tuple containing a state and mismatch counters $S(s,i,d,a)$, and a mismatch type $M$, the $t$-$reachable$ of S is a 4 tuple with the state that is reachable following the tagged transtion of the mismatch type $M$ in $\Delta'$ and the mismatch counters, with the mismatch type increased by one.  
\end{mydef}
\begin{algorithm}[h!]
  \caption{TNFA Simulation
    \label{tnfasim}}
  \begin{algorithmic}[1]
    \Require{$N$ is a TNFA and $s$ is a string,M is a 3-tuple of mismatches allowed}
    \Ensure{True if $s$ is accepted in $N$ with $M$ mismatches, False if $s$ is rejected in $N$}
    \Function{TNFA Simulation}{$N(Q,\Sigma,\Delta,q^s,q^a,\Delta'), s,M$}
      \Let{$stateset$}{$\{q^s,0,0,0\}$} 
      \For{{\bf each} symbol {\bf in} $s$}
        \If{$stateset = \emptyset$} 
            \State \Return{False}
        \EndIf
        \Let{$next$}{$\emptyset$}
        \Let{$states$}{$\epsilon$-closure($stateset$)}
        \Let{$next$}{$reachable(symbol,states)$}
        	\Let{$t\_next$}{$TNFA$-$reach(states,M)$}
          \Let{$stateset$}{$next \cup t\_next$}
      \EndFor
      \If{$q^a \in stateset$}
        \State \Return{True}
      \EndIf
      \State \Return{False}
    \EndFunction
  \end{algorithmic}
\end{algorithm}

\begin{algorithm}[h!]
  \caption{
    \label{tnfatrans}}
  \begin{algorithmic}[1]
    \Require{$states$ is a set of 4 tuples with a state q and mismatches occured, $M$  is a 3-tuple of mismatches allowed}
    \Ensure{A stateset with states that are reachable from $states$ using mismatches}
    \Function{$TNFA$-$reach$}{$states,M(ins,del,alt)$}
      \Let{$stateset$}{$\emptyset$} 
      \For{{\bf each} state(s,i,d,a) {\bf in} $states$}
        \Let{$stateset$}{$stateset \cup reachable(state)$}
        \If{$i < ins$}
        	\Let{$stateset$}{$stateset ~ \cup ~ (t$-$reachable(state,i))$}
        \EndIf
        \If{$d < del$}
        	\Let{$stateset$}{$stateset ~\cup ~ (t$-$reachable(state,d))$}
        \EndIf
        \If{$a < alt$}
        	\Let{$stateset$}{$stateset ~\cup~ (t$-$reachable(state,a))$}
        \EndIf
      \EndFor
      \State \Return{$stateset$}
    \EndFunction
  \end{algorithmic}
\end{algorithm}

\newpage


\begin{myex}
Given the RE $E= {\tt abc}^*{\tt d}$ the resulting TNFA N can be seen in Figure \ref{tnfa:simsuc}
\begin{figure}[H]
\begin{center}
\begin{tikzpicture}[->,>=stealth,shorten >=1pt,auto,node distance=3 cm, scale = 0.75, transform shape,initial text={}]
  \node [initial, state] (0) {0};
  \node [state,right of=0] (1) {1};
  \node      [state,right of=1] (2) {2};
   \node     [accepting,state,right of=2] (3) {3};
    \node    [state,above left of=2,yshift=1cm] (4) {4};
    \node    [state,above right of=2,yshift=1cm] (5) {5};


   \path[->] (0) edge node {{\tt a}} (1)
         (1) edge node {{\tt b}} (2)
         (2) edge node {{\tt d}} (3)
           edge node [near end] {$\epsilon$} (4)
         (4) edge node {{\tt c}} (5)
         (5) edge node [near start] {$\epsilon$} (2);


  \path[->] (0) edge [color=green, in=100,out=80,loop] node [color=black, above] {$\epsilon/i$} (0);
  \path[->] (0) edge [color=red,bend left] node [color=black, above] {$\epsilon/d$} (1);
  \path[->] (0) edge [color=blue,bend right] node [color=black, below] {$\epsilon/a$} (1);


  \path[->] (1) edge [color=green, in=100,out=80,loop] node [color=black, above] {$\epsilon/i$} (1);
  \path[->] (1) edge [color=red,bend left] node [color=black, above] {$\epsilon/d$} (2);
  \path[->] (1) edge [color=blue,bend right] node [color=black, below] {$\epsilon/a$} (2);

  \path[->] (2) edge [color=green, in=100,out=80,loop] node [color=black, above] {$\epsilon/i$} (2);
  \path[->] (2) edge [color=red,bend left] node [color=black, above] {$\epsilon/d$} (3);
  \path[->] (2) edge [color=blue,bend right] node [color=black, below] {$\epsilon/a$} (3);

  \path[->] (4) edge [color=green, in=100,out=80,loop] node [color=black, above] {$\epsilon/i$} (4);
  \path[->] (4) edge [color=red,bend left] node [color=black, above] {$\epsilon/d$} (5);
  \path[->] (4) edge [color=blue,bend right] node [color=black, below] {$\epsilon/a$} (5);
\end{tikzpicture}
\end{center}
\caption{TNFA of expression $E = {\tt abc}^*{\tt d}$}
\label{tnfa:simsuc}
\end{figure}
We now want to see if the input string {\tt "abbdd"} is accepted in N allowing 1 insertion and 1 deletion. The initial stateset \{($q^s$,i,d,a)\} =\{(0,0,0,0)\}\\\\
\begin{table}[h!]
\small
\begin{tabular}{l l l}
\multicolumn{2}{l}{TNFASimulate(N,{\tt abbdd},(1,1,0)):}\\
symbol {\tt a}:& $\epsilon$-$closure$(\{(0,0,0,0)\})&=\{(0,0,0,0)\}\\
&$reachable$(\{(0,0,0,0)\},{\tt a})&=\{(1,0,0,0)\}\\
&$t$-$reachable$(\{(0,0,0,0)\},i)&=\{(0,1,0,0)\}\\
&$t$-$reachable$(\{(0,0,0,0)\},d)&=\{(1,0,1,0)\}\\
&$next\_stateset$&=\{(1,0,0,0),(0,1,0,0),(1,0,1,0)\}\\
\\
symbol {\tt b}:& $\epsilon$-$closure$(\{(1,0,0,0),(0,1,0,0),(1,0,1,0)\}) &=\{(1,0,0,0),(0,1,0,0),(1,0,1,0)\}\\
&$reachable$(\{(1,0,0,0),(0,1,0,0),(1,0,1,0)\},{\tt b})&=\{(2,0,0,0),(2,0,1,0)\}\\
&$t$-$reachable$(\{(1,0,0,0),(0,1,0,0),(1,0,1,0)\},i)&=\{(1,1,0,0),(1,1,1,0)\}\\
&$t$-$reachable$(\{(1,0,0,0),(0,1,0,0),(1,0,1,0)\},d)&=\{(2,0,1,0),(1,1,1,0)\}\\
&$next\_stateset$&=\{(2,0,0,0),(2,0,1,0),(1,1,0,0),(1,1,1,0)\}\\
\\ 
symbol {\tt b}:&$\epsilon$-$closure$(\{(2,0,0,0),(2,0,1,0),(1,1,0,0),(1,1,1,0)\})&=\{(2,0,0,0),(2,0,1,0),(1,1,0,0)\\
&&~~~,(1,1,1,0),(4,0,0,0),(4,0,1,0)\}\\
&$reachable$(\{(2,0,0,0),(2,0,1,0),(1,1,0,0),(1,1,1,0)&=\{(2,1,0,0),(2,1,1,0)\}\\
&~~~~~~~~~~~~~~,(4,0,0,0),(4,0,1,0)\},{\tt b})\\

&$t$-$reachable$(\{(2,0,0,0),(2,0,1,0),(1,1,0,0),(1,1,1,0)&=\{(2,1,0,0),(2,1,1,0),(4,1,0,0),(4,1,1,0)\}\\
&~~~~~~~~~~~~~~~~,(4,0,0,0),(4,0,1,0)\},i)\\

&$t$-$reachable$(\{(2,0,0,0),(2,0,1,0),(1,1,0,0),(1,1,1,0)&=\{(3,0,1,0),(3,1,1,0),(5,0,1,0)\}\\
&~~~~~~~~~~~~~~~~,(4,0,0,0),(4,0,1,0)\},d)\\

&$next\_stateset$&=\{(2,1,0,0),(2,1,1,0),(3,0,1,0),(3,1,1,0)\\
&&~~~,(4,1,0,0),(4,1,1,0),(5,0,1,0)\}\\
\\
symbol {\tt d}:&$\epsilon$-$closure$(\{(2,1,0,0),(2,1,1,0),(3,0,1,0),(3,1,1,0)&=\{(2,1,0,0),(2,0,1,0),(2,1,1,0),(3,0,1,0),(3,1,1,0)\\
&~~~~~~~~~~~~~~,(4,1,0,0),(4,1,1,0),(5,0,1,0)\}&~~~,(4,1,0,0),(4,0,1,0),(4,1,1,0),(5,0,1,0)\}\\
&$reachable$(\{(2,1,0,0),(2,0,1,0),(2,1,1,0),(3,0,1,0),(3,1,1,0)&=\{(3,1,0,0),(3,0,1,0),(3,1,1,0)\}\\
&~~~~~~~~~~~~~~,(4,1,0,0),(4,0,1,0),(4,1,1,0),(5,0,1,0)\},{\tt d})\\

&$t$-$reachable$(\{(2,1,0,0),(2,0,1,0),(2,1,1,0),(3,0,1,0),(3,1,1,0)&=\{(3,1,1,0),(5,1,1,0)\}\\
&~~~~~~~~~~~~~~~~,(4,1,0,0),(4,0,1,0),(4,1,1,0),(5,0,1,0)\},i)\\

&$t$-$reachable$(\{(2,1,0,0),(2,0,1,0),(2,1,1,0),(3,0,1,0),(3,1,1,0)&=\{(3,1,1,0),(5,1,1,0)\}\\
&~~~~~~~~~~~~~~~~,(4,1,0,0),(4,0,1,0),(4,1,1,0),(5,0,1,0)\},d)\\
&$next\_stateset$&=\{(3,1,0,0),(3,0,1,0),(3,1,1,0),(5,1,1,0)\}\\
\\
symbol {\tt d}:&$\epsilon$-$closure$(\{(3,1,0,0),(3,0,1,0),(3,1,1,0),(5,1,1,0)\})&=\{(3,1,0,0),(3,0,1,0),(3,1,1,0),(5,1,1,0)\\
&&~~~,(2,1,1,0),(4,1,1,0)\}\\
&$reachable$(\{(3,1,0,0),(3,0,1,0),(3,1,1,0),(5,1,1,0)&=\{3,1,1,0\}\\
&~~~~~~~~~~~~~~,(2,1,1,0),(4,1,1,0)\},{\tt d})\\
&$t$-$reachable$(\{(3,1,0,0),(3,0,1,0),(3,1,1,0),(5,1,1,0)&=$\emptyset$\\
&~~~~~~~~~~~~~,(2,1,1,0),(4,1,1,0)\},i)\\
&$t$-$reachable$(\{(3,1,0,0),(3,0,1,0),(3,1,1,0),(5,1,1,0)&=$\emptyset$\\
&~~~~~~~~~~~~~,(2,1,1,0),(4,1,1,0)\},d)\\
&$final\_stateset$&=\{3,1,1,0\}
\end{tabular}
\end{table}\\
The string {\tt "abbdd"} is accepted, since $q^a\in final\_stateset$, which have 1 insertion and 1 deletion.
\end{myex}
\begin{myex}
Given the RE $E={\tt abbca}$ the TNFA $N$ of $E$ is seen in Figure \ref{tnfa:simfail}.
\begin{figure}[h!]
\begin{center}
\begin{tikzpicture}[->,>=stealth,shorten >=1pt,auto,node distance=3 cm, scale = 0.75, transform shape,initial text={}]
  \node [initial, state] (0) {0};
  \node [state,right of=0] (1) {1};
  \node      [state,right of=1] (2) {2};
   \node     [state,right of=2] (3) {3};
   \node     [state,right of=3] (4) {4};
   \node     [accepting,state,right of=4] (5) {5};


   \path[->] (0) edge node {{\tt a}} (1)
         (1) edge node {{\tt b}} (2)
         (2) edge node {{\tt b}} (3)
         (3) edge node {{\tt c}} (4)
         (4) edge node {{\tt a}} (5);


  \path[->] (0) edge [color=green, in=100,out=80,loop] node [color=black, above] {$\epsilon/i$} (0);
  \path[->] (0) edge [color=red,bend left] node [color=black, above] {$\epsilon/d$} (1);
  \path[->] (0) edge [color=blue,bend right] node [color=black, below] {$\epsilon/a$} (1);


  \path[->] (1) edge [color=green, in=100,out=80,loop] node [color=black, above] {$\epsilon/i$} (1);
  \path[->] (1) edge [color=red,bend left] node [color=black, above] {$\epsilon/d$} (2);
  \path[->] (1) edge [color=blue,bend right] node [color=black, below] {$\epsilon/a$} (2);

  \path[->] (2) edge [color=green, in=100,out=80,loop] node [color=black, above] {$\epsilon/i$} (2);
  \path[->] (2) edge [color=red,bend left] node [color=black, above] {$\epsilon/d$} (3);
  \path[->] (2) edge [color=blue,bend right] node [color=black, below] {$\epsilon/a$} (3);

  \path[->] (3) edge [color=green, in=100,out=80,loop] node [color=black, above] {$\epsilon/i$} (3);
  \path[->] (3) edge [color=red,bend left] node [color=black, above] {$\epsilon/d$} (4);
  \path[->] (3) edge [color=blue,bend right] node [color=black, below] {$\epsilon/a$} (4);

  \path[->] (4) edge [color=green, in=100,out=80,loop] node [color=black, above] {$\epsilon/i$} (4);
  \path[->] (4) edge [color=red,bend left] node [color=black, above] {$\epsilon/d$} (5);
  \path[->] (4) edge [color=blue,bend right] node [color=black, below] {$\epsilon/a$} (5);
\end{tikzpicture}
\end{center}
\caption{TNFA of expression $E = {\tt abbca}$}
\label{tnfa:simfail}
\end{figure}
We now want to see the string {\tt "abccec"} is accepted into $N$ given the initial stateset \{(0,0,0,0)\} and allowing 1 insertion and 1 alternation.\\
\begin{table}[h!]
\small
\begin{tabular}{l l l}
\multicolumn{2}{l}{TNFASimulate(N,{\tt abccec},(1,0,1)):}\\
symbol {\tt a}:& $\epsilon$-$closure$(\{(0,0,0,0)\})&=\{(0,0,0,0)\}\\
&$reachable$(\{(0,0,0,0)\},{\tt a})&=\{(1,0,0,0)\}\\
&$t$-$reachable$(\{(0,0,0,0)\},i)&=\{(0,1,0,0)\}\\
&$t$-$reachable$(\{(0,0,0,0)\},a)&=\{(1,0,0,1)\}\\
&$next\_stateset$&=\{(1,0,0,0),(0,1,0,0),(1,0,0,1)\}\\
\\
symbol {\tt b}:& $\epsilon$-$closure$(\{(1,0,0,0),(0,1,0,0),(1,0,0,1)\})&=\{(1,0,0,0),(0,1,0,0),(1,0,0,1)\}\\
&$reachable$(\{(1,0,0,0),(0,1,0,0),(1,0,0,1)\},{\tt b})&=\{(2,0,0,0),(2,0,0,1)\}\\
&$t$-$reachable$(\{(1,0,0,0),(0,1,0,0),(1,0,0,1)\},i)&=\{(1,1,0,0),(1,1,0,1)\}\\
&$t$-$reachable$(\{(1,0,0,0),(0,1,0,0),(1,0,0,1)\},a)&=\{(1,1,0,1),(2,0,0,1)\}\\
&$next\_stateset$&=\{(1,1,0,0),(1,1,0,1),(2,0,0,0),(2,0,0,1)\}\\
\\
symbol {\tt c}:& $\epsilon$-$closure$(\{(1,1,0,0),(1,1,0,1),(2,0,0,0),(2,0,0,1)\})&=\{(1,1,0,0),(1,1,0,1),(2,0,0,0),(2,0,0,1)\}\\
&$reachable$(\{(1,1,0,0),(1,1,0,1),(2,0,0,0),(2,0,0,1)\},{\tt c})&=$\emptyset$\\
&$t$-$reachable$(\{(1,1,0,0),(1,1,0,1),(2,0,0,0),(2,0,0,1)\},i)&=\{(2,1,0,0),(2,1,0,1)\}\\
&$t$-$reachable$(\{(1,1,0,0),(1,1,0,1),(2,0,0,0),(2,0,0,1)\},a)&=\{(2,1,0,1),(3,0,0,1)\}\\
&$next\_stateset$&=\{(2,1,0,0),(2,1,0,1),(3,0,0,1)\}\\
\\
symbol {\tt c}:& $\epsilon$-$closure$(\{(2,1,0,0),(2,1,0,1),(3,0,0,1)\})&=\{(2,1,0,0),(2,1,0,1),(3,0,0,1)\}\\
&$reachable$(\{(2,1,0,0),(2,1,0,1),(3,0,0,1)\},{\tt c})&=\{(4,0,0,1)\}\\
&$t$-$reachable$(\{(2,1,0,0),(2,1,0,1),(3,0,0,1)\},i)&=\{(3,1,0,1)\}\\
&$t$-$reachable$(\{(2,1,0,0),(2,1,0,1),(3,0,0,1)\},a)&=\{(3,1,0,1)\}\\
&$next\_stateset$&=\{(3,1,0,1),(4,0,0,1)\}\\
\\
symbol {\tt e}:& $\epsilon$-$closure$(\{(3,1,0,1),(4,0,0,1)\})&=\{(3,1,0,1),(4,0,0,1)\}\\
&$reachable$(\{(3,1,0,1),(4,0,0,1)\},{\tt e})&=$\emptyset$\\
&$t$-$reachable$(\{(3,1,0,1),(4,0,0,1)\},i)&=\{(4,1,0,1)\}\\
&$t$-$reachable$(\{(3,1,0,1),(4,0,0,1)\},a)&=$\emptyset$\\
&$next\_stateset$&=\{(4,1,0,1)\}\\
\\
symbol {\tt c}:& $\epsilon$-$closure$(\{(4,1,0,1)\})&=\{(4,1,0,1)\}\\
&$reachable$(\{(4,1,0,1)\},{\tt c})&=$\emptyset$\\
&$t$-$reachable$(\{(4,1,0,1)\},i)&=$\emptyset$\\
&$t$-$reachable$(\{(4,1,0,1)\},a)&=$\emptyset$\\
&$next\_stateset$&=$\emptyset$\\
\\
\end{tabular}
\end{table}\\\\\\
At the end of reading input $\emptyset$ is found, the string {\tt abccec} is not accepted into $N$.
\end{myex}