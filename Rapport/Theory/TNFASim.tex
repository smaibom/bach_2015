
\subsection{Simulating TNFA}\label{section_TNFA}
TNFA simulation adds an additional argument for amount of mismatches allowed. The stateset is a 4 tuple of $(Q,i,d,a)$ where i,d and a is a count of how many of each transition has been used. $\epsilon-closure$ and $reachable$ transfers this count over to each of the states reached in these functions.
\begin{mydef}
$tagreach/s$ 
\end{mydef}
\begin{algorithm}[h!]
  \caption{TNFA simulation
    \label{nfasim}}
  \begin{algorithmic}[1]
    \Require{$N$ is a TNFA and $x$ is a string,M is a 3-tuple of mismatches allowed}
    \Function{Simulation}{$N(Q,\Sigma,\Delta,q^s,q^a,\Delta'), x,M$}
      \Let{$stateset$}{$\{q^s\}$} 
      \For{{\bf each} symbol {\bf in} $x$}
        \If{$stateset = \emptyset$} 
            \State \Return{False}
        \EndIf
        \Let{$next$}{$\emptyset$}
        \Let{$states$}{$\epsilon$-closure($stateset$)}
        \Let{$states$}{$reachable(symbol,states)$}
        	\Let{$next$}{$TNFA-trans(states,M)$}
          \Let{$stateset$}{$next$}
      \EndFor
      \If{$q^a \in stateset$}
        \State \Return{True}
      \EndIf
      \State \Return{False}
    \EndFunction
  \end{algorithmic}
\end{algorithm}

\begin{algorithm}[h!]
  \caption{
    \label{nfasim}}
  \begin{algorithmic}[1]
    \Require{$\Delta'$ is a tagged transition table, states is a set of 4 tuples with a state q and mismatches occured, M is a 3-tuple of mismatches allowed}
    \Function{TNFA-trans}{$states,M(ins,del,alt)$}
      \Let{$stateset$}{$\emptyset$} 
      \For{{\bf each} state(s,i,d,a) {\bf in} $states$}
        \Let{$stateset$}{$stateset \cup reachable(state)$}
        \If{$i < ins$}
        	\Let{$stateset$}{$stateset ~ \cup ~ (tagreach(state,I)$}
        \EndIf
        \If{$d < del$}
        	\Let{$stateset$}{$stateset ~\cup ~ (tagreach(state,D))$}
        \EndIf
        \If{$a < alt$}
        	\Let{$stateset$}{$stateset ~\cup~ (tagreach(state,A))$}
        \EndIf
      \EndFor
      \State \Return{$stateset$}
    \EndFunction
  \end{algorithmic}
\end{algorithm}

\newpage


\begin{myex}
Given the RE $E= {\tt abc}^*{\tt d}$ the resulting TNFA N can be seen in Figure \ref{tnfa:simsuc}
\begin{figure}[h!]
\begin{center}
\begin{tikzpicture}[->,>=stealth,shorten >=1pt,auto,node distance=3 cm, scale = 0.75, transform shape,initial text={}]
  \node [initial, state] (0) {0};
  \node [state,right of=0] (1) {1};
  \node      [state,right of=1] (2) {2};
   \node     [accepting,state,right of=2] (3) {3};
    \node    [state,above left of=2,yshift=1cm] (4) {4};
    \node    [state,above right of=2,yshift=1cm] (5) {5};


   \path[->] (0) edge node {{\tt a}} (1)
         (1) edge node {{\tt b}} (2)
         (2) edge node {{\tt d}} (3)
           edge node [near end] {$\epsilon$} (4)
         (4) edge node {{\tt c}} (5)
         (5) edge node [near start] {$\epsilon$} (2);


  \path[->] (0) edge [color=green, in=100,out=80,loop] node [color=black, above] {$\epsilon/i$} (0);
  \path[->] (0) edge [color=red,bend left] node [color=black, above] {$\epsilon/d$} (1);
  \path[->] (0) edge [color=blue,bend right] node [color=black, below] {$\epsilon/a$} (1);


  \path[->] (1) edge [color=green, in=100,out=80,loop] node [color=black, above] {$\epsilon/i$} (1);
  \path[->] (1) edge [color=red,bend left] node [color=black, above] {$\epsilon/d$} (2);
  \path[->] (1) edge [color=blue,bend right] node [color=black, below] {$\epsilon/a$} (2);

  \path[->] (2) edge [color=green, in=100,out=80,loop] node [color=black, above] {$\epsilon/i$} (2);
  \path[->] (2) edge [color=red,bend left] node [color=black, above] {$\epsilon/d$} (3);
  \path[->] (2) edge [color=blue,bend right] node [color=black, below] {$\epsilon/a$} (3);

  \path[->] (4) edge [color=green, in=100,out=80,loop] node [color=black, above] {$\epsilon/i$} (4);
  \path[->] (4) edge [color=red,bend left] node [color=black, above] {$\epsilon/d$} (5);
  \path[->] (4) edge [color=blue,bend right] node [color=black, below] {$\epsilon/a$} (5);
\end{tikzpicture}
\end{center}
\caption{TNFA of expression $E = {\tt abc}^*{\tt d}$}
\label{tnfa:simsuc}
\end{figure}
We now want to see if the input string {\tt "abbcdd"} is accepted in N allowing 1 insertion and 1 deletion.
\begin{tabular}{l l l}
TNFASimulate(N,{\tt abbcdd},\{1,1,0\}) & & \\
symbol {\tt a}:& $\epsilon$-$closure$(\{(0,0,0,0)\}) & = \{(0,0,0,0)\}\\
& tagreach(\{(0,0,0,0)\},a) & = \{(0,1,0,0),(1,0,0,0),(1,0,1,0)\}\\
symbol {\tt b}:& $\epsilon$-$closure$(\{(0,1,0,0),(1,0,0,0),(1,0,1,0)\}) & = \{(0,1,0,0),(1,0,0,0),(1,0,1,0)\}\\
& tagreach(\{(0,1,0,0),(1,0,0,0),(1,0,1,0)\},b) & = \{(1,1,1,0),(1,1,0,0),(2,0,0,0)\}\\
symbol {\tt b}:& $\epsilon$-$closure$(\{(1,1,1,0),(1,1,0,0),(2,0,0,0)\}) &= \{(1,1,1,0),(1,1,0,0),(2,0,0,0),(4,0,0,0)\}\\
& tagreach(\{(1,1,1,0),(1,1,0,0),(2,0,0,0),(4,0,0,0)\}) = \{(1,1,1,0),(2,1,1,0),(2,1,0,0),(3,1,1,0)
(1,1,0,0),(2,0,0,0)\}
\end{tabular}
\end{myex}
\{(1,1,1,0),(1,1,0,0),(2,0,0,0)\}