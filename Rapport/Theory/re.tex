\section{Regular expressions} 
  To explain what a regular expression is, we must first introduce languages and alphabets. All literals will be written using the typewriter font, to distinguish between literals and other text. 
\begin{mydef}\label{alph}
An alphabet $\Sigma$ is a finite non-empty set of letters.
\end{mydef}

\begin{mydef}\label{lang}
A language L is a subset of all strings formed over $\Sigma:~ L\subseteq \Sigma^*$
\end{mydef}

\begin{myex}
If we have a DNA sequence string, the alphabet $\Sigma$ consists of the literals ${\tt\{t,g,c,a\}}$. and the language contains strings formed by the literals from this alphabet. For example "gtcaaa" or "gtcaaat". 
\end{myex}


\begin{mydef}
A regular expression is described by the following grammar: \\
\begin{center}
$E::= a|0|1|E_1 + E_2 |E_1 E_2 | E^*$
\end{center}
where $E_1$ and $E_2$ are RE's and $a \in \Sigma$
\end{mydef}

\begin{mydef}\label{re}
The language interpation L(E) of a regular expression is: 
\begin{center}
\begin{align*}
L(0)           &= \emptyset\\
L(1)         &= \{\epsilon\} \\
L({\tt a})     &= \{{\tt a}\} \\
L(E_0 + E_1) &= L(E_0) \cup L(E_1) \\
L(E_1 E_2)   &= \{w_1w_2 | w_1 \in L(E_1),w_2 \in L(E_2)\}=L(E_1)L(E_2) \\
L(E^*)       &= \bigcup\limits_{n=0}^\infty L(E)^n \\
L(E)^0       &= \{\epsilon\}\\
L(E)^n       &= \underbrace{L(E)L(E)...L(E).}_\text{n times}
\end{align*}
\end{center}
\cite[p.5 def. 3]{crash}
\end{mydef}

With definition \ref{re}, we can now form regular languages. 
\begin{myex}Natural numbers can be described as a regular expression. Natural numbers have the alphabet $\Sigma$ = \{{\tt 0,1,2,3,4,5,6,7,8,9}\}, the regular expression for natural numbers would look like:
\begin{center}
$E_{nat} = $(1+2+3+4+5+6+7+8+9)(0+1+2+3+4+5+6+7+8+9)$^*$
\end{center}
\end{myex}