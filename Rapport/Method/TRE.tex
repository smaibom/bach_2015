\section{TRE}\label{tre}
  
For our implementation of scan\_for\_matches patterns in TRE, we needed to 
first make a brief analysis of the program to see what it could and couldn't 
do. What we discovered after our initial tests was that TRE defined each 
new line as a delimiter (a delimiter defines at which points the text will be 
separated, so each chunk can be evaluated individually). 
This caused a small, but surmountable, problem; chromosome data 
should be read as one continous line that spans over many lines. Since TRE 
split the data up per line, if a match spanned over more than one line, it 
wouldn't find it. To this problem there was an easy fix; we would make a small 
wrapper which would feed the text file to TRE, removing any newlines as they 
occurred.\\
With this problem solved, we did a trial test to check the accuracy and 
running time of TRE. Here we encountered a second, greater problem; TRE 
would find only one match per delimiter (the first longest match with fewest 
errors, prioritizing exact matches). This was a harder problem to fix, because this feature was 
deeply entrenched in the code, and everything had seemingly been designed 
around this feature. The searching tool that TRE uses, agrep, only receives 
the best match that the other tools of TRE has found, but throughout TRE's 
code, it won't save the matches it has found, but instead repeatedly discard old 
matches. The problem is compounded by the fact that TRE will immediately stop searching the 
text if it finds an exact match, leaving behind potential matches. In light of 
these problems, we decided that attempting to fully understand TRE's design and 
subsequently heavily modding it in order for it to return all matches would 
take too long, and that we would instead create our own solution.
