%a previously found match (backreferencing)
%a modified pattern (reverse, complementary)
\section{TRE}\label{tre}
Recall in Section~\ref{probanal}, we determined the challenge of our 
solution would lie in matching:
\begin{enumerate}
\item \label{err} with alterations allowed
\item \label{backr} a previously found match 
\item \label{rev} a modified pattern
\end{enumerate}
An implementation based on TRE would address two of these three 
problems. Item~\ref{err} would require approximate matching support, which 
TRE supports. Item~\ref{backr} could be resolved 
with backreferences (even if backreferences are computationally inefficient as 
they are an NP-complete problem, which also makes them not regular) 
which TRE partly support. Item~\ref{rev} means that sometimes we want to 
match a modification of a previously found pattern - like the reverse 
complement of a pattern. This would have to be implemented in TRE's parser 
(to denote a symbol for the modified patterns, e.g. the reverse or the 
complement of a pattern) as 
well as in TRE's basic functionality.\\\\
When analyzing TRE so we could start modifying the 
program, we discovered that TRE would define every newline as a delimiter. The 
delimiter specifies how the data should split, causing every new line 
to be loaded into the buffer and processed. This in 
itself would not be a problem if TRE did not discard any match currently being 
processed, causing matches wrapping around two 
lines to be discarded. The fix to this could be to make a wrapper 
was created which would feed the text data to TRE, 
ignoring all newlines, then TRE would load the entire file into its buffer, 
which would no longer cause it to discard potential matches across newlines.\\\\
Upon running a file through TRE, ignoring newlines, we discovered 
another feature of TRE, which would not work for our project; TRE would only 
find one match per delimiter, it being the earliest, best-matching match (where a 
best-matching match is a match with the least amount of errors). Since we ignored 
newlines, TRE would only 
return one match. TRE was built around this feature, which led to the following 
design choices;
\begin{itemize}
\item the program runs through the data once to determine how many errors the 
best-matching match has, and then runs through the text file again, stopping 
when the best-matching match has been found 
\item TRE ignores any matches that are not best-matching, meaning there is 
no way for TRE to identify and output acceptable matches.
\end{itemize}
This coupled with limited documentation on the code meant it would 
take a long time to properly analyze TRE in order to find out how we could 
modify it to suit our goals. At this point we decided that creating our 
own solution would be less time-consuming, while also allowing us to design our 
solution ourselves.
