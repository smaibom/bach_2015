%a previously found match (backreferencing)
%a modified pattern (reverse, complementary)
\section{TRE}\label{tre}
Recall that in section~\ref{probanal}, we determined that the challenge of our 
solution would lie in matching:
\begin{enumerate}
\item \label{err} with errors allowed,
\item \label{backr} a previously found match, and 
\item \label{rev} a modified pattern.
\end{enumerate}
An implementation based on TRE would address two of the three previously mentioned 
problems. Item~\ref{err} would require approximate matching support, and 
TRE fully supports approximate matching. Item~\ref{backr} could be resolved 
with backreferences (even if backreferences are computationally inefficient and not regular) 
and TRE also supports this. Item~\ref{rev} means that sometimes we want to 
match a modification of a pattern found previously - like the reverse 
complement of a pattern. This would have to be implemented in TRE's parser 
(to denote a symbol for the modified patterns, e.g. the reverse or the 
complement of a pattern) as 
well as in TRE's basic functionality.\\\\
When analyzing TRE so we could start modifying the 
program, we discovered that TRE would define every newline as a delimiter. The 
delimiter specifies how the data should be broken up, so for every new line 
the current line would be loaded into the buffer and be processed. This in 
itself would not be a problem if not TRE would discard any match that 
was currently being processed when it reached a delimiter, causing matches that 
wrapped around two lines to be discarded. The fix to this was easy to make 
however; if a wrapper was created which would feed the text data to TRE, 
ignoring all newlines, then TRE would load the entire file into its buffer 
and would no longer cause it to discard potential matches that continued to 
the next line.\\\\
When we then tried to run a file through TRE which had no newlines, we discovered 
another feature of TRE which would not work for our project; TRE would only 
match one match per delimiter - the earliest, best-matching match (where a 
best-matching match is a match with the least amount of errors). Since we had 
to trim the text files for TRE so there were no newlines, TRE would only 
return one match. TRE was built around this feature, which led to the following 
design choices;
\begin{itemize}
\item the program runs through the data once to determine how many errors the 
best-matching match has, and then runs through the text file again, stopping 
when the best-matching match has been found, 
\item TRE ignores any matches that are not best-matching, meaning there is 
no way for TRE to identify and output acceptable matches.
\end{itemize}
This coupled with little documentation of how the code worked meant that it would 
take a long time to properly analyze TRE in order to find out how we could 
modify it to suit our project. At this point we decided that creating our 
own solution would be less time-consuming, while also allowing us to design our 
solution ourselves.
