\section{Introduction}%Probably write some about what we did, and 
%What our result(s) were?
When the Human Genome Project~\cite{hgp} (a project which had the goal of sequencing 
all 22 chromosomes of the human genome) was launched in 1990, the project was 
budgeted to cost 3 billion dollars and was estimated to take fifteen years 
to complete. However as technology progressed, the project managed to complete 
its goal two years earlier than expected, in 2003. This was made possible 
because of the rapid advancements in genome sequencing, and the advancement 
has not stopped since. This has led to decreasing costs of sequencing RNA and DNA, 
meaning biologists has access to greater amounts of data than before. 
However the technology to process these amounts of data have not progressed at 
the same pace as sequencing. Scan\_for\_matches is a tool for pattern-matching, 
which searches through data files to match a pattern specified by a user. 
While scan\_for\_matches has proven to be a fast and reliable 
tool, due to the amount of data it shifts through, a faster alternative 
is desired.\\\\
In this thesis, we provide an alternative to scan\_for\_matches based on 
automata theory and regular expressions. While the implementation currently 
does not match scan\_for\_matches' speed, with optimization it will. We will 
discuss the implementation's strengths and weaknesses, and describe what 
future work with the implementation will involve.
Our implementation can be found at 
"\url{https://github.com/smaibom/bach_2015/tree/master/Implementation/src}".
\begin{comment}
After hearing about this problem, we thought that there must be a better 
way of searching through data that is also theoretically sound. Our first 
thought was using automata-based searching methods, since this provides a 
calculable best- and worst-case run time while being theoretically sound. 
Since regular expressions uses an automata-based way of searching, we hypothesized 
that implementing regular expressions which have the same functions as 
scan\_for\_matches would lead to faster run times.
\end{comment}
