\documentclass[11pt,twoside,a4paper]{article}

\begin{document}
\section{Considered Solutions}
\subsection{Kleene Meets Church}
Kleene Meets Church (KMC) is a regex-based markup engine being developed by 
the KMC group in DIKU, headed by Fritz Henglein, professor at DIKU. The KMC 
engine aims at providing a faster processing time than other regex engines, 
while making it easier for non-computer scientists to use.\\
As our project synopsis states, we first wanted to use KMC to implement 
scan\_for\_matches. This was because we had heard of 
how fast the KMC engine was, so it seemed like the logical engine to use. 
However we turned to other solutions when we found out that 
KMC is intended only for text-markup, meaning it doesn't return the position 
of a match or the match itself. 

\subsection{TRE}
A regex engine that we considered for our implementation of scan\_for\_matches 
was TRE. TRE is an open-source engine implemented in C by Ville Laurikari for 
his master's thesis in 2001 with the goal to implement approximate matching in 
regex. By using a mix of Levenshtein distancing (which describes 
approximate matching) and tagged NFAs (tNFA), Laurikari accomplished this with a 
linear complexity of O(MN), where M is the size of a regular expression and N is the 
size of the input string\cite{LaurikariComplex}, which means that it maintains the 
regex standard complexity even with approximate matching. Our original plan 
for TRE was to create a wrapper which would serve two purposes:
\begin{itemize}
\item Format the input for TRE, 
\item Convert scan\_for\_matches patterns to TRE-style regex, and
\item Format the output for TRE
\end{itemize}
While investigating our possible solutions with TRE, a few problems 
with the implementation meant that we 
had to look for another solution. The first problem we found with TRE was that 
it used the POSIX.2 regex standard. This meant that TRE does not support the 
global pattern modifier, so TRE would only return the earliest best-matching 
match per delimiter\cite{Best-Match}. This combined with another problem we found 
- TRE is unable to match a string which spans over a delimiter - meant that 
TRE would require heavy modification for it to work how we wanted it to. Given our 
limited time and the insufficient documentation in the code meant that 
we had to look for another solution.

\begin{thebibliography}{9}
\bibitem{LaurikariComplex}
Ville Laurikari.
\textit{Efficient submatch addressing for regular expressions, section 3.3.2}
2001

\bibitem{Best-Match}
\textit{A best-matching match is a match with the lowest Levenshtein Distance among other matches}

\end{thebibliography}

\end{document}
